\documentclass[a4paper,11pt]{memoir}

\title{Running a Fencing Tournament Using the Fencing Time Program}
\author{Colin Walls}
\date{1st January 2013}

\usepackage{color,calc,graphicx,soul,url,fourier}
\usepackage[utf8]{inputenc}
\usepackage[nonumberlist]{glossaries}

\usepackage{tikz}
\usetikzlibrary{calc}

\definecolor{nicered}{rgb}{.647,.129,.149}
\makeatletter
\newlength\dlf@normtxtw
\setlength\dlf@normtxtw{\textwidth}
\def\myhelvetfont{\def\sfdefault{mdput}}
\newsavebox{\feline@chapter}
\newcommand\feline@chapter@marker[1][4cm]{%
\sbox\feline@chapter{%
\resizebox{!}{#1}{\fboxsep=1pt%
\colorbox{nicered}{\color{white}\bfseries\sffamily\thechapter}%
}}%
\rotatebox{90}{%
\resizebox{%
\heightof{\usebox{\feline@chapter}}+\depthof{\usebox{\feline@chapter}}}%
{!}{\scshape\so\@chapapp}}\quad%
\raisebox{\depthof{\usebox{\feline@chapter}}}{\usebox{\feline@chapter}}%
}
\newcommand\feline@chm[1][4cm]{%
\sbox\feline@chapter{\feline@chapter@marker[#1]}%
\makebox[0pt][l]{% aka \rlap
\makebox[1cm][r]{\usebox\feline@chapter}%
}}
\makechapterstyle{daleif1}{
\renewcommand\chapnamefont{\normalfont\Large\scshape\raggedleft\so}
\renewcommand\chaptitlefont{\normalfont\huge\bfseries\scshape\color{nicered}}
\renewcommand\chapternamenum{}
\renewcommand\printchaptername{}
\renewcommand\printchapternum{\null\hfill\feline@chm[2.5cm]\par}
\renewcommand\afterchapternum{\par\vskip\midchapskip}
\renewcommand\printchaptertitle[1]{\chaptitlefont\raggedleft ##1\par}
}
\makeatother
\chapterstyle{daleif1}

\makeglossary
\newglossaryentry{club}{name=club, description={A club is any organized group of fencers that has established themselves as a club. Fencers who are not members of a particular club are said to be “unattached”.}}
\newglossaryentry{competition}{name=competition, description={A competition is one of the event that form part of a tournament, e.g. Under-11 boys foil}}
\newglossaryentry{competitor}{name=competitor, description={A competitor is someone taking part in one of the events in a tournament}}
\newglossaryentry{csv file}{name=CSV file, description={CSV stands for “comma-separated value” and refers to a file format commonly generated by spreadsheet programs. Such files may contain lists of competitors, clubs, or ranking points. \fencingtime{} can import these files and also export them for other purposes.}}
\newglossaryentry{database}{name=database, description={A database is a collection of related data stored by a program. \fencingtime{} stores fencers, clubs, and referees in its master database. This is done so that the same fencers, clubs, and referees can participate in multiple tournaments without having to be re-entered each time.}}
\newglossaryentry{event}{name=event, description={An event is another name for a competition.}}
\newglossaryentry{indicators}{name=indicators, description={These are used to calculate the ranking after a seeding round, there are three of them. The first is the number of victories divided by the number of matches (V/M). Fencers with equal V/M ratios are separated by the number of hits (touches) they have scored minus the number of hits they have received (HS - HR, or TS – TR, depending on whether one is using British or American terminology). If fencers are still equal after this then the final separation is on hits scored (HS or TS). If fencers are still equal then their ranking is assigned randomly.}}
\newglossaryentry{membership number}{name=membership number, description={This is the term used in \fencingtime{} for a fencer's membership in the national organisation. In the UK this corresponds to the BFA or Home Country membership number.}}
\newglossaryentry{piste}{name=piste, description={The field of play for a fencing bout, it is 14m long and 1.5 to 2m wide. All tournaments will have a number of pistes.}}
\newglossaryentry{ranking}{name=ranking, description={A fencer's ranking is their position on a points list. The fencer with the most points is rank 1; the fencer with the second most points is rank 2, and so on. \fencingtime{} supports multiple rankings per fencer since they could have different rankings on different point lists and for different weapons. Rankings are used when seeding an event along with the fencer's weapon rating.}}
\newglossaryentry{round}{name=round, description={A round is a “part” of an event. There are several kinds of rounds; the round of pools and the direct elimination round are the two most common. Typically, a round will seed the competitors for the next round.}}
\newglossaryentry{strip}{name=strip, description={The American term for a piste.}}
\newglossaryentry{seed}{name=seed, description={A seed is a competitor's current position in an event. At the start of the event, competitors are usually seeded by rating with the highest rated competitors seeded first and the lowest rated seeded last. As the event progresses, competitors will be re-seeded based on their performance in each round.}}
\newglossaryentry{tournament}{name=tournament, description={A tournament is made up of one or more events. For example, the British Youth Championships consist of  foil, \'{e}p\'{e}e and sabre events for for under-10, -12, -14, -16 and under-18 fencers.}}
\newglossaryentry{directoire technique}{name=directoire technique,description={The Directoire Technique has jurisdiction over all the fencers who take part in or are present at a competition which it is running. It is responsible for maintaining order and discipline during tournaments and may activate any penalties specified in the Rules for Competitions. Its decisions are immediately enforceable and not subject to appeal during a tournament}}
\newglossaryentry{dt}{name=DT, description={The abbreviation for the the Directoire Technique}}
\newglossaryentry{referee}{name=referee, description={All fencing bouts are controlled by the referee. He makes the initial call for the fencers, checks their weapons, clothing and equipment. He directs the bout, maintaining order as well as awarding hits and penalties.}} 
\newglossaryentry{bout}{name=bout, description={An assault at which the score is kept. Usually refers to a match between two fencers in a competition.}}
\newglossaryentry{black card}{name=black card, description={A black card is used to indicate the most serious offences in a fencing competition. The offending fencer is expelled immediately from the event or tournament, regardless of whether he or she had any prior warnings. A black card can also be used to expel a third party disrupting the match.}}
    
\begin{document}
\frontmatter % Front matter begins

\begin{titlingpage}
\newlength{\drop}
\definecolor{Dark}{gray}{0.2}
\begin{center}
\drop=0.1\textheight
\fboxsep 0.5\baselineskip
\sffamily
\vspace*{\drop}
\centering
{\textcolor{nicered}{\HUGE Running a Fencing Tournament}}\par
\vspace{0.5\drop}
\colorbox{Dark}{\textcolor{white}{\normalfont\itshape\Large
Using the Fencing Time Program}}\par
\vspace{\drop}
{\Large Colin Walls}\par
\vfill
{\Large January 2013}\par
\vfill
\includegraphics[width=6cm,keepaspectratio]{./Logos/britishFencing}
\par
{\footnotesize British Fencing Association}\par
\vspace*{\drop}
\includegraphics[width=6cm,keepaspectratio]{./Logos/bfnwr}
\vfill
\end{center}
\end{titlingpage}

% Common look and feel for the program name
\newcommand{\fencingtime}{\emph{{\color{nicered}{Fencing Time}}}}

% Rounded boxes for buttons
\newcommand\button[2][]{\tikz[overlay]\node[fill=blue!20,inner sep=2pt, anchor=text, rectangle, rounded corners=1mm,#1] {#2};\phantom{#2}}

% Square boxes for menu items
\newcommand\menu[2][]{\tikz[overlay]\node[fill=gray!20,inner sep=2pt, anchor=text, rectangle ,#1] {#2};\phantom{#2}}

\section*{Licence}
\begin{center}
\includegraphics[keepaspectratio]{./Logos/by-sa} 
\end{center}

This document is licensed under the Creative Commons Attribution -- Share Alike 3.0 Unported (CC BY-SA 3.0) licence (
\url{https://creativecommons.org/licenses/by-sa/3.0/}).

You are free:

\begin{itemize}
 \item to copy, distribute, display, and perform the work
 \item to make derivative works
 \item to make commercial use of the work
\end{itemize}

Under the following conditions:

\begin{description}
 \item[Attribution] You must attribute the work in the manner specified by the author or licensor (but not in any way that suggests that they endorse you or your use of the work).
 \item[Share Alike] If you alter, transform, or build upon this work, you may distribute the resulting work only under the same or similar license to this one.
\end{description}

With the understanding that:

 \begin{description}
  \item[Waiver] Any of the above conditions can be \textbf{waived} if you get permission from the copyright holder.
  \item[Public Domain] Where the work or any of its elements is in the \textbf{public domain} under applicable law, that status is in no way affected by the license. 
  \item[Other Rights] In no way are any of the following rights affected by the license:
  \begin{enumerate}
   \item Your fair dealing or \textbf{fair use} rights, or other applicable copyright exceptions and limitations;
   \item The author's \textbf{moral} rights;
   \item Rights other persons may have either in the work itself or in how the work is used, such as \textbf{publicity} or privacy rights.
  \end{enumerate}
 \end{description}

\tableofcontents

\listoffigures

\listoftables

\graphicspath{{./screenCaptures/}}

% Main matter begins
\mainmatter

\chapter{Introduction}

This document describes how to run a fencing tournament using the \fencingtime{} software. This includes:
\begin{itemize}
 \item Installing the software
 \item Adding clubs, fencers and referees to the database
 \item Setting up the tournament
 \item Adding the events
 \item Importing entries and rankings
 \item Running Poule and Direct Elimination rounds
 \item Exporting results for the use on web sites and by points coordinators
\end{itemize}

\section{Conventions Used in this Manua}
The followed typographical conventions are used throughout this document:

\begin{tabular}{p{2.5cm}p{9cm}}
 \menu{menu} & This indicates either a menu (such as \menu{File}) or a part of a menu (such as \menu{Save}). When you need to select an item from a particular menu this will be indicated by an arrow, e.g. \menu{File} $\rightarrow$ \menu{Save} \\
 \button{button} & This indicates a button in the user interface that should be clicked upon to activate it, e.g. \button{OK} \\
 \texttt{Text} & This indicates part of the user interface such as labels on text entry boxes or tabs.
\end{tabular}

\chapter{Installing Fencing Time}

\section{Getting a Copy of the Program}

The program is available as a free download from the \fencingtime{} web site \url{http://www.fencingtime.com/FencingTime/}. However this only works for competitions limited to seven fencers or less. To run a larger tournament you will need to licence the program.

\section{Set Up}

The downloaded file is a Windows executable, double clicking on it will take you through the install process. 

\section{Licensing the Program}

When you first run \fencingtime{}, you will see a screen indicating that it is running in ``Trial Mode'', while this will allow you to run \fencingtime{} it has a number of restrictions on what you can do. 

\begin{figure}[!ht]
 \centering
 \includegraphics[width=6cm,keepaspectratio]{02-introductoryScreen}
 \caption{Introductory Screen}\label{fig:02-introductoryScreen}
\end{figure}

You can continue using the program by clicking on the \button{Continue} button. This will allow you to run the program in the restricted trial mode.

To purchase a licence click on the \button{Buy Fencing Time Now!} button. This will take you to the \fencingtime{} web site and guide you through purchasing a licence.

If you have purchased a key but not yet licenced the program then click on the \button{Enter Licence Key Now!} button. You should then copy and paste the key from the email that has been sent to you and uniquely identifies you as the purchaser of the software. It is very important that you keep the email with the license key so that if you ever need to install \fencingtime{} on a different computer, you can re-use the key to unlock the full version. You should also be sure to keep your license key secret and not share it with others.

\section{Configuration}
The program requires very little initial configuration, the only change that is necessary is to reset the default country. Select the  \menu{Tools} $\rightarrow$ \menu{Preferences} menu item. On the \texttt{Fencing} tab select \texttt{Great Britain} from the drop down list as shown in figure \ref{fig:02-defaultCountry}.

\begin{figure}[!ht]
  \centering
  \begin{minipage}{0.4\textwidth}
    \centering
    \includegraphics[width=4cm,keepaspectratio]{02-fencingPreferencesWindow}
    \caption{Setting the Default Country}\label{fig:02-defaultCountry}
  \end{minipage}
  \hfill
  \begin{minipage}{0.4\textwidth}
    \centering
    \includegraphics[width=4cm,keepaspectratio]{02-printingPreferencesWindow}
    \caption{Setting the Printing Preferences} \label{fig:02-printingPreferences}
  \end{minipage}
\end{figure}

It is also possible to select a custom logo to be included on all printed output. From the same window select the \texttt{Printing} tab and select the image that you wish to use, it should be no more than 200x100 pixels in size.

\section{Customising Displayed and Printed Information}
Some default information displayed on the screen or on printed output such as divisions or classifications is only relevant to American tournaments. In addition you may wish to limit information shown because it is not appropriate for your tournament or simply because your screen size is limited.

\begin{figure}[!ht]
 \centering
 \includegraphics[width=10cm,keepaspectratio]{02-fullEntriesWindow}
 \caption{Uncustomised Window} \label{fig:02-uncustomisedWindow}
\end{figure}

\fencingtime{} will allow you to customise the content of virtually all displayed and printed information. If one has a screen similar to that shown in figure \ref{fig:02-uncustomisedWindow} then the \texttt{Division} and \texttt{Rating} columns are for American tournaments and are not required. You may not wish to have the \texttt{Born} column printed on the check in or competitors sheet. If you press the right mouse button whilst the cursor is on the title bar for the column then you will be presented with a drop down list allowing you to select the columns that you wish to have displayed. After deselecting the \texttt{Division}, \texttt{Rating} and \texttt{Born} windows the above screen now looks like that shown in figure \ref{fig:02-customisedWindow}.

\begin{figure}[!ht]
 \centering
 \includegraphics[width=10cm,keepaspectratio]{02-trimmedEntriesWindow}
 \caption{Customised Window} \label{fig:02-customisedWindow}
\end{figure}

\section{Caveats}
\fencingtime{} is a flexible and easy to use program for running tournaments. However the current version of the software is very much geared to the American fencing community and the \gls{ranking} and \gls{competition} structure in place there. In most cases this will not be a problem, in places where the user should avoid the default action because it applies specifically to the way American tournaments are run then this will be indicated in the instructions.

\chapter{The User Interface}

\begin{figure}[!ht]
 \centering
 \includegraphics[width=12cm,keepaspectratio]{03-mainScreen}
 \caption{The Main Screen} \label{fig:03-mainScreen}
\end{figure}

The main window for \fencingtime{} is broken into four parts shown by red arrows in figure \ref{fig:03-mainScreen}.

\begin{enumerate}
 \item The main menu bar
 \item The main toolbar
 \item The left side panel
 \item The events window
\end{enumerate}

\section{Main Menu Bar}
At the top of the window is a menu bar with the standard drop-down menus that you see in most other programs. 

\begin{itemize}
 \item The \menu{File} menu provides commands to create new tournaments and access \gls{tournament} files saved on your disk. 
 \item The \menu{View} menu provides access to various additional screens in \fencingtime{}, as well as different reports that summarize tournament information. 
 \item The \menu{Tools} menu offers a number of helpful commands that you can use to configure \fencingtime{}. 
 \item The \menu{Windows} menu lets you organize the event windows you have open.
 \item The \menu{Help} menu provides access to various reference materials and guides.
\end{itemize}

\section{Main Toolbar}
The toolbar immediately below the main menu bar provides you with buttons that serve as short cuts to many functions that are found on the main menu bar, as well as a number of useful ways to quickly enter data into \fencingtime{}.

\section{Left Side Panel}
On the left side of the window is the left side panel. The content of this panel varies depending on whether you are currently running a tournament or not. When no tournament is loaded, the panel will contain buttons that let you create or load tournaments. Once a tournament is loaded (as shown in the picture), the panel will display a list of the events in the tournament and buttons used to control them.

The left menu panel can be hidden from view by clicking the blue arrow button near the top. This will collapse it into a small bar along the left edge of the window. Clicking the arrow again will return the panel to its normal size.

\section{Event windows}
The largest part of the \fencingtime{} screen is area where the event windows are shown. Each event in your tournament appears in its own window here when you choose to view it from the list of events. You can arrange the windows here as you choose.

\chapter{Preliminaries -- Setting up the Database}
When running an event the entries are drawn from \fencingtime{}'s database rather than individual entries having to be typed in. While this is an initial overhead it pays dividends in consequent years since it avoids having to re-enter information for fencers as you run new tournaments. 

This chapter mainly deals with the initial bulk import of data into the database. Once you have this you should rarely need to do it again, instead modifications will usually consist of the addition of individual entries to the database and modification of entries that are already there. The only exception to this will be rankings, these change constantly and for many fencers hence bulk import will be the normal way of updating these.

There are three parts to the database:

\begin{enumerate}
 \item Clubs
 \item Fencers, this includes ranking information
 \item Referees
\end{enumerate}

Entries for each of these parts may be done by importing a CSV file or by typing into a data entry window.

\section{Clubs} 
Since both fencers and referees will be associated with clubs then it is probably best to set the clubs list up first. 

\subsection{Importing a Club List}
Use the \menu{File} $\rightarrow$ \menu{Close} menu item to make sure all tournaments are closed. Then select \menu{View} $\rightarrow$  \menu{Clubs Database}. This will produce the window shown in figure \ref{fig:04-clubsDatabase}.

\begin{figure}[!ht]
 \centering
 \includegraphics[width=10cm,keepaspectratio]{04-clubsDatabase}
 \caption{The Main Clubs Window} \label{fig:04-clubsDatabase}
\end{figure}

Click on the \button{Import Clubs} button, this will produce the window shown in figure \ref{fig:04-clubsImport} allowing you to select the type of import you wish to make.

\begin{figure}[!ht]
 \centering
 \includegraphics[width=5cm,keepaspectratio]{04-clubsImport}
 \caption{The Clubs Import Window} \label{fig:04-clubsImport}
\end{figure}

You should select the \texttt{Any other CSV file} option and click the \button{Next} button. This will produce a standard file selection window allowing to to choose the file to import, this should be a \gls{csv file} containing, at a minimum, a header giving the name of the entries, and a list of clubs one per line. Optionally each line may also contain an abbreviation separated from the name by a comma. A typical file would look like figure \ref{fig:04-clubsCSV}.

\begin{figure}[!ht]
 \begin{verbatim}
Name,Abbreviation
West Hill School,WHSFC
Stockport Grammar School,SGSFC
 \end{verbatim}
 \caption{A Typical Clubs CSV File} \label{fig:04-clubsCSV}
\end{figure}

Once the file has been opened you will be presented with a window containing your field headers and what field in the database these correspond to. If you wish to assign a field then double click on the \texttt{Ignore this column} next to it, this will produce a drop down list from which you can select. The only field that you must select is the \texttt{Club Name}. All other fields can be ignored.

\begin{figure}[!ht]
 \begin{minipage}{0.4\textwidth}
  \centering
  \includegraphics[width=5cm,keepaspectratio]{04-clubFields}
  \caption{Fields for Club Import} \label{fig:04-clubFields}
 \end{minipage}
 \hfill
 \begin{minipage}{0.4\textwidth}
  \centering
  \includegraphics[width=5cm,keepaspectratio]{04-newClub}
  \caption{New and Edit Clubs Dialogue} \label{fig:04-newClub}
 \end{minipage}
\end{figure}

Once you have selected the fields then click on the \button{Next} button to import the clubs into the database. A log of the actions will be produced, watch for any error messages that indicate one or more clubs have not been imported. These can be added individually as necessary, either as the clubs part of the database is being set up or during the import of fencers for a competition.

\subsection{Adding Individual Clubs}
If there are errors in importing clubs, or only a small number of clubs need to be added to the database then this is most easily done using the \button{New Club} function on the clubs database main window. Clicking on this button will produce the data entry window shown in figure \ref{fig:04-newClub}.

The only fields that you must fill in are the one for the \texttt{Club name} and the \texttt{Abbreviation}. All the other fields are optional and can be safely ignored. Click on the \button{OK} button to save the details.

\subsection{Editing a Club}

If there are errors in the club details then clicking on the edit button on the extreme left of the clubs database main window (the icon that looks like a pencil writing on a pad) will produce the same window shown in figure \ref{fig:04-newClub} for a new club but with the fields filled in. You should change the details that are in error and then click on the \button{OK} button to save the modified information.

\section{Fencers}\label{sec:fencers}
Adding fencers to the database is very similar to adding clubs. As before make sure all the tournaments are closed by using the \menu{File} $\rightarrow$ \menu{Close Tournament} menu item, then use the \menu{View} $\rightarrow$ \menu{Fencer Database} menu item to open the fencer database window shown in figure \ref{fig:04-fencerDatabase}.

\begin{figure}
 \centering
 \includegraphics[width=8cm,keepaspectratio]{04-fencerDatabase}
 \caption{The Main Fencer Database Window} \label{fig:04-fencerDatabase}
\end{figure}

This window can be used for searching the database for fencers using criteria, such as surname or \gls{membership number}. It will also display details of fencers matching those criteria and allow the records for these fencers to be edited. 

It also allows the creation of new fencers and bulk operations, such as the import of new fencers and ranking lists.

\subsection{Importing a List of Fencers}

If you are collecting entries for tournaments online and you have an epty database then you may be able to import your data directly into \fencingtime{}.  Export your entries to a CSV file containing at least the following information:

\begin{enumerate}
 \item Forename
 \item Surname
 \item Primary club
 \item Year of birth
 \item Gender
 \item BFA number
\end{enumerate}

Where there may be a confusion in forenames, e.g. Matt and Matthew or Tom and Thomas, then you might also consider including a nickname. 
A typical CSV file might look like that shown in figure \ref{fig:04-fencersCSV}.

\begin{figure}[!ht]
 \begin{verbatim}
BFA_Number,Forename,Surname,Gender,Year_of_Birth,Primary_club
850291,Owen,Edwards,M,1994,Wimbledon Fencing Club
599815,David,Croker,M,1993,Dunes Fencing Club
856029,Michael,Taylor,M,1981,Guernsey Fencing Club
558328,Jan,Thomas,F,1956,Newham Swords
388176,Maureen,Edwards,F,1990,Gravesham Fencing Club
508229,Terry,Marvin,M,1994,Maidstone Fencing Club
 \end{verbatim}
 \caption{A Typical Fencers CSV File} \label{fig:04-fencersCSV}
\end{figure}

Once you have your CSV file then click on the \button{Import Fencers} button which will produce the screen shown in figure \ref{fig:04-fencerImport}. You should select the \texttt{None of the above} option and press the \button{Next} button.

\begin{figure}[!ht]
 \centering 
 \begin{minipage}{0.4\textwidth}
  \centering
  \includegraphics[width=5cm,keepaspectratio]{04-fencerImport}
  \caption{Selecting the Fencer CSV File Type} \label{fig:04-fencerImport}
 \end{minipage}
 \hfill
 \begin{minipage}{0.4\textwidth}
  \centering
  \includegraphics[width=5cm,keepaspectratio]{04-fencerFields}
  \caption{Associating CSV and Database Fields} \label{fig:04-fencerFields}
 \end{minipage}
\end{figure}

This will produce the window shown in figure \ref{fig:04-fencerFields}, to select which fields in the file correspond to fields in the \fencingtime{} database double-click on \texttt{Ignore this column} next to the field you want to associate. This will change to a drop down list from which you can select.  

As has been noted above you should enter the forename, surname, primary club, year of birth, gender and BFA number for the fencer. If the fencer is not affiliated to a club then leave this field blank in the CSV file and the affiliation will default to \texttt{Unattached}.  You should not put ``Unattached'' in the CSV file. 

Once you have selected all the fields then click on the \button{Next} button and your data will be imported. Review the log that is produced to ensure that there are no errors. If some fencers have not been properly imported then these can be added individually or details edited to ensure that they are correct. You should also check that fencers have not been given an ``Unattached'' affiliation because their club is not in the database, or is there under a different name, e.g. ``LTFC'' instead of ``London Thames Fencing Club''. 

\subsection{Using the BFA Membership Database as the Fencer Database}

An alternative to using online entries or preparing your own CSV file would be to use the BFA membership database. This is available on application from BFA Head Office. However the data extract only contains the forename, surname and BFA number. The year of birth, gender and primary club are missing from the extract and need to be added manually.

\subsection{Adding Individual Fencers}
If there are errors when you import a set of fencers, or you are only adding a small number then it is easier to add these using a data entry window. Click on the \button{New Fencer button} in the main fencer window shown in figure \ref{fig:04-fencerDatabase}. This will produce the dialogue shown in figure \ref{fig:04-newFencer}.

\begin{figure}[!ht]
 \centering
 \includegraphics[width=6cm,keepaspectratio]{04-newFencer}
 \caption{New and Edit Fencer Dialogue} \label{fig:04-newFencer}
\end{figure}

In the \texttt{Personal Data} panel you should fill in the last and first names and the year of birth if known (this is important for age group competitions since \fencingtime{} will check to ensure that fencers meet the criterion for the competition). As has been indicated above, where a first name may have an abbreviation then it may well be worthwhile including a nickname. \texttt{The Membership \#} field should be filled in with the BFA number, EFC or FIE licence as appropriate.

In the \texttt{Affiliation} panel you should fill in the primary club. If the club is not in the drop down list you can add a new one by clicking on the \button{New} button. This will take you to the new club dialogue shown in figure \ref{fig:04-newClub}. If the new fencer is British then the remainder of the fields can be disregarded. If the new fencer fencer is foreign then you should set the country of origin.

The bottom panel labelled \texttt{Classifications} will be dealt with in the section on rankings.

\subsection{Rankings}\label{sec:rankings}

This section forms the major difference between the American and British systems of ranking fencers. Particular attention should be paid to the process of accurately matching fencers to their ranks. 

\fencingtime{} allows you to have multiple ranking lists, you should create one or more appropriate to the kind of tournament you are running. For events which are are mixed (such as age group competitions) you will only need to create a single list. For tournaments which include competitions for athletes of a particular gender then you will need to create a list for each.

Select \menu{View} $\rightarrow$ \menu{Point Lists}, then the \texttt{Custom Point Lists} tab. Click on the \button{New List} button to create a new list and fill in the details, you must specify the name of the list, an abbreviation and the organisation to which it applies. Selecting an age category will assist in checking whether competitors meet the requirements for the tournament.

\begin{figure}[!ht]
 \centering
 \begin{minipage}{0.4\textwidth}
 \centering
 \includegraphics[width=5cm,keepaspectratio]{04-managePointsLists}
 \caption{Custom Points List} \label{fig:04-managePointsLists}  
 \end{minipage}
 \hfill
 \begin{minipage}{0.4\textwidth}
 \centering
 \includegraphics[width=5cm,keepaspectratio]{04-newPointsList}
 \caption{Creating a New Points List} \label{fig:04-newPointsList}    
 \end{minipage}
\end{figure}

Note that you do not need to create a new list for each weapon since \fencingtime{} gives each fencer a ranking for all three weapons.

\subsubsection{Importing Point Ranks}

As with other parts to the database \fencingtime{} will allow you to import a CSV file. Wherever you get your information from it needs to contain:

\begin{description}
 \item[A method of identifying the fencer] ideally this should be the BFA number but \fencingtime{} will also try to match on the surname, forename and nickname.
 \item[Ranking information] this may either be a position on the ranking table or the number of points the fencer has earned in previous competitions. In general it is best to use the position, but in certain circumstances this may not be possible. For example, figure \ref{fig:04-rankingsCSV} shows a typical ranking file, as can be seen two entries are for foreign fencers who do not have positions. In this case one would use the points values and end up with a relative rather than absolute ranking.
\end{description}

\begin{figure}[!ht]
 \begin{verbatim}
Position,Forename,Surname,Points
1,Bob,Pepper,17452
2,Rich,Curtis,9899
3,David,Appleton,9236
4,Doug,Richmond,8988
5,Peter,Parkinson,8532
NED,Harry,Dekock,8217
6,Pat,Farrell,7834
7,Bill,Ward,7733
8,Paul,Swaffield,7421
9,David,Arthur,7366
10,Arthur,Weightman,7204
CH,Peter,Landrebe,7103
11,Pat,Sinclair,6982
12,Bill,Strang,6719 
 \end{verbatim}
 \caption{A Typical Rankings CSV File} \label{fig:04-rankingsCSV}
\end{figure}

Select \button{Import Point Ranks} from the main fencer database window shown in figure \ref{fig:04-fencerDatabase}. This will give you the usual file browser dialogue enabling you to choose your CSV file. Once you have loaded this then you should select the fields  in the usual way, as is shown in figure \ref{fig:04-pointsListsFields}. You must choose sufficient fields to identify the fencer and the ranking. Once you have done this then click on the \button{Next} button to import the data. You will be prompted as to whether the list you are importing is complete or partial. You should normally chose \texttt{Yes, this is the complete point list} since this will replace all the ranking points. Choosing the \texttt{No, this is only a partial point list} will only update the fencers in the imported list, other fencers will not have their ranking points changed.

\begin{figure}
 \centering
 \begin{minipage}{0.4\textwidth}
 \centering
 \includegraphics[width=5cm,keepaspectratio]{04-pointsListsFields}
 \caption{Selecting Point List Fields} \label{fig:04-pointsListsFields}  
 \end{minipage}
 \hfill
 \begin{minipage}{0.4\textwidth}
 \centering
 \includegraphics[width=5cm,keepaspectratio]{04-importPointRanksComplete}
 \caption{Importing a Complete Points List} \label{fig:04-importPointRanksComplete}    
 \end{minipage}
\end{figure}

A log will be produced showing the which ranks were accepted and which were not. Where the program is unable to determine which fencer the rank applies to it will leave a gap. This can be amended by hand either before the competition or when it has been started. 

\section{Referees}
\fencingtime{} also keeps a section of its database for referees. If you plan to allocate referees on a formal basis rather than simply handing score sheets to any available \gls{referee} then you will need a list of referees to assign.

It is possible to import referees into the database using a CSV file in the same way as fencers or clubs. However unlike fencers or clubs the CSV file has to to be in a specific, and complex, format. Because of this, and because the number of referees will be small compared to the number of fencers it is suggested that the list is generated from the referees dialogue window.

\begin{figure}[!ht]
 \centering
 \begin{minipage}{0.4\textwidth}
 \centering
 \includegraphics[width=5cm,keepaspectratio]{04-refereeMainScreen}
 \caption{The Main Referee Screen} \label{fig:04-refereeMainScreen}  
 \end{minipage}
 \hfill
 \begin{minipage}{0.4\textwidth}
 \centering
 \includegraphics[width=5cm,keepaspectratio]{04-newRefereeScreen}
 \caption{The New and Edit Referee Dialogue} \label{fig:04-newRefereeScreen}    
 \end{minipage}
\end{figure}

Ensure that any open tournament is closed by selecting \menu{File} $\rightarrow$ \menu{Close Tournament} and then from the \menu{View} $\rightarrow$ \menu{Referee Database}. This will produce the window shown in figure \ref{fig:04-refereeMainScreen}.

Click on the \button{New Referee} button to produce the window shown in figure \ref{fig:04-newRefereeScreen}. You should fill in at least the \texttt{Last name} and \texttt{First name} fields as well as the individual weapon ratings. Where a referee is affiliated with a club you should fill this in too, this will assist \fencingtime{} to avoid allocating referees to poules or direct elimination bouts where there could be a possible conflict because of a fencer and a referee coming from the same club.

Once you have filled the details in then press the \button{OK} button to save the details.

\chapter{Planning the Tournament}

There are a number of things that need to be taken into account when planning a tournament, these include
\begin{description}
 \item[Venue] you will need to ensure that it has sufficient space for the number of people who will attend, including car parking space. Its location is also a consideration, if it is difficult to get to then this may put possible competitors off. If you are running a tournament in a location which is not permanently open, such as a school, then you will need to ensure that someone is available to open the venue in good time both for laying of pistes on the days of the tournament.
 \item[Availability of accommodation] though you may not be responsible for accommodation a venue which has good, local hotels and boarding houses will make your tournament more attractive.
 \item[Catering] fencing tournaments are all day affairs. If your tournament is in a venue which does not have catering then you should either make this known when you advertise the tournament, consider employing outside caterers or getting volunteers from your club to provide catering. When you consider the number of fencers and those accompanying them this can make a good contribution to the overall profit.
 \item[Equipment] you will obviously need scoring equipment for each \gls{piste} plus some spares. For larger competitions sponsored by equipment vendors then this may not be a problem, but for smaller tournaments you may have to borrow from local clubs if your own club does not have sufficient equipment. Remember that you will need tables both for scoring boxes and as a good place for piste numbers . You should also consider whether conductive pistes are necessary, whether they should be used throughout the tournament and just for the finals.
 \item[Marketing] it used to be the case that all one needed to do to market a tournament was to place an advertisement in \emph{The Sword} which included an entry form. These days most tournaments of any size will either have a dedicated web site or a section on a club web site, often with the ability to do online entry. Web sites for tournaments should also include details of the programme, location of the venue with directions (Google Maps is ideal for this), tournament contacts and possibly a downloadable entry form. Other things that add interest are results from previous years, photo galleries and links to social media such as Facebook and Twitter. 
 \item[Entry forms] if you do have a web site for a tournament then you should consider online entry. There are a number of event registration applications for standard website software, writing a registration application is a major undertaking and should not be undertaken by anyone without a great deal of software development experience. Regardless of whether you use an online entry system or paper entry then you should determine what information you need to collect, you should not collect more information than is necessary. The following is fairly typical:
 \begin{enumerate}
  \item Identification information, this should include forename and surname as well as appropriate licence details
  \item Contact information, this should include phone numbers (land line and mobile) and email address. This is required in order to inform competitors of any changes to schedules
  \item Competition information, this may include the weapon, gender and age group
 \end{enumerate}
 \item[Safety] fencing is a safe sport and accidents are rare. For a large tournament you may wish to consider using an emergency first aid organisation such as St. John's Ambulance if the venue does not provide dedicated first aiders. A risk assessment of the premises may be a useful exercise.
\end{description}

A major constraint in all of this is of course your budget!  Something that will not be considered further in this document.

\section{Timing and Piste Usage}
You will almost certainly not be able to provide enough pistes to run all your events simultaneously.  You will therefore need to ensure that all your pistes have the highest possible usage.  To assist you in doing this table \ref{tab:05-estimatedTimings} gives some rule of thumb estimates for the time to run a \gls{round}.

\begin{table}[!ht]
 \centering
 \begin{tabular}{rlll}
  \toprule
  & Foil & \'{E}p\'{e}e & Sabre \\
  \midrule
  Poule of 6 & 1.5 hours & 1.5 hours & 1 hour \\
  Poule of 7 & 2 hours & 2 hours & 1.5 hours \\
  DE bout (15 hits) & 15 minutes & 15 minutes & 10 minutes \\
  Team Match & 1 hour & 1 hour & 45 minutes \\
  \bottomrule
 \end{tabular}
 \caption{Estimated Timings for a Round} \label{tab:05-estimatedTimings}
\end{table}

You should allow additional time for the following:

\begin{itemize}
 \item From check in closing until the start of the competition
 \item From the end of the poule round to the start of the DE
 \item Between each phase of the DE
\end{itemize}

In each case a minimum of 30 minutes should be allowed.

Besides maximising the use of pistes you should also aim to minimise the amount of time that fencers have to wait between rounds, this is especially true for incomplete tableaux where some fencers will have byes.

A simple tool for planning piste usage is a spreadsheet, a typical example is shown in figure \ref{fig:05-tournamentPlanningSheet}. The one disadvantage is that it is static, which makes it difficult to track which pistes are in use, which fights are allocated to which pistes, what time score sheets went out and whether they have been returned or not. A large whiteboard with the tournament plan on it can help with this.

\begin{figure}[!ht]
 \centering
 \includegraphics[width=12cm,keepaspectratio]{05-tournamentPlanningSheet}
 \caption{Spreadsheet for Tournament Planning} \label{fig:05-tournamentPlanningSheet}
\end{figure}

Another alternative, for those with the experience and the software, would be a project planning application such as Microsoft Project.

\section{Referees}

The type of plan outlined above depends not just on the number of pistes you have, but the number of referees as well. It is of no use to schedule a round on eight pistes when you only have four referees. You obviously need to make best use of your referees without overworking them.

If you have a number of referees then you will need to coordinate their usage with the DT. This will mean designating one of them as a lead referee and ensuring that you have a helper who can pass information on referee allocation to the DT.

\section{Helpers}

Tracking the running of the tournament is the key to success. If you know exactly what is going on at all times then you can ensure that any problems are dealt with immediately and not allowed to escalate. 

You will need a group of volunteers who are allocated specific tasks, these include:

\begin{description}
 \item[Check in desk] ideally one person who will check people off on the list and a second to take any outstanding entry fees. If you have multiple competitions checking in at the same time then you should have one person per check in list if possible.
 \item[Runners] to liaise between the \gls{dt}, referees and floor managers. 
 \item[Referees' assistants] to liaise between the \gls{directoire technique} and referees
 \item[Floor managers] to ensure that the equipment is working properly and also to make sure that bouts that are supposed to be running are actually running
 \item[Welfare staff]
 \item[Armourers] For larger tournaments armoury services may be supplied by the equipment vendor who is attending the event. For smaller events you will need to provide at least sufficient armourer involvement to make sure that pistes are running properly.
\end{description}

\section{Directiore Technique}

The DT is a vital part of any tournament, however small. You should ensure that you use the most experienced people you can find.
The role of the DT varies with the size of the tournament. For a large tournament its principle role is to manage fencing related matters, rather than the operational side of the tournament. For smaller competitions the two roles may be combined.

However, you should ensure that there are sufficient members of the DT available to resolve any issues and disputes without effecting the smooth running of the tournament.

Some of the tasks carried out by the DT will include:

\begin{itemize}
 \item Agreeing the formula for the tournament with the competition organiser, this will include the number of rounds of poules, whether there should be a cut after the poules, whether repêchage should take place etc.
 \item Agreeing the timetable, including the number of pistes and allocation of referees.
 \item Deciding on any re-drawing of the poules in response to a successful appeal.
 \item Resolving any dispute in written paperwork (incorrect score sheets).
 \item Deciding on the course of action to be taken in response to a fencer scratching or being excluded.
\end{itemize}

This is a non-exhaustive list, other tasks will almost certainly be required.

\chapter{Before the Tournament}
\section{Creating the Tournament}

Close any open tournaments and click the \button{New Tournament} button on the left panel shown in in figure \ref{fig:06-newTournament} to create the new tournament.

\begin{figure}
 \centering
 \includegraphics[width=5cm,keepaspectratio]{06-newTournament}
 \caption{Creating a New Tournament} \label{fig:06-newTournament}
\end{figure}

This will bring up the window shown in figure \ref{fig:06-initialTournamentScreen}, fill in the name of the tournament, its location and the dates on which it will take place. Unless you are going to charge to actually register for the tournament then ignore the \texttt{Registration fee} box, a later screen will allow you to set an entry fee. Click on the \button{Next} button, this will take you to the initial screen of the Event Wizard, where you will be taken through a series of screens to set up the first event of the tournament.

\begin{figure}
 \centering
 \includegraphics[width=5cm,keepaspectratio]{06-initialTournamentScreen}
 \caption{Creating a New Tournament} \label{fig:06-initialTournamentScreen}
\end{figure}

\section{The New Event Wizard}

This wizard is used not only to create the first event of a tournament, but all other competitions that make up the tournament as well.

The initial screen is shown in figure \ref{fig:06-eventWizardStep-1}, fill in the type of competition (individual or team) and the weapon that is to be fenced (foil, épée or sabre) and press the \button{Next} button.

\begin{figure}[!ht]
 \centering
 \begin{minipage}{0.4\textwidth}
  \centering
  \includegraphics[width=5cm,keepaspectratio]{06-eventWizardStep-1}
  \caption{Event Wizard -- Step 1, the Event Type} \label{fig:06-eventWizardStep-1}
\end{minipage}
\hfill
 \begin{minipage}{0.4\textwidth}
  \centering
  \includegraphics[width=5cm,keepaspectratio]{06-eventWizardStep-2}
  \caption{Event Wizard -- Step 2, Date, Time and Cost} \label{fig:06-eventWizardStep-2}
\end{minipage}
\end{figure}

\section{Date, Time and Cost}

Once you have have defined the type of competition you need to specify when it will take place and what the entry fee is. This is done in the window shown in figure \ref{fig:06-eventWizardStep-2}.

\section{Gender, Age and Classification}

The next three screens allow you set the gender, age and classification of the fencers taking part in the tournament. Only the first two figures are appropriate for competitions in Britain. You should be aware though that direct elimination bout for the youth age groups are fenced as three poule bouts rather than as a single \gls{bout} with a smaller number of hits. 

\begin{figure}[!ht]
 \centering
 \begin{minipage}{0.4\textwidth}
  \centering
  \includegraphics[width=5cm,keepaspectratio]{06-eventWizardStep-3}
  \caption{Event Wizard -- Step 3, Gender} \label{fig:06-eventWizardStep-3}
\end{minipage}
\hfill
 \begin{minipage}{0.4\textwidth}
  \centering
  \includegraphics[width=5cm,keepaspectratio]{06-eventWizardStep-4}
  \caption{Event Wizard -- Step 4, Age Range} \label{fig:06-eventWizardStep-4}
\end{minipage}
\end{figure}

The final screen in this series allows you to select the classification of fencers for the event, this is purely for American tournaments and you should always select \texttt{None}.

\begin{figure}[!ht]
 \centering
 \includegraphics[width=5cm,keepaspectratio]{06-eventWizardStep-5}
 \caption{Event Wizard -- Step 5, American Classifications} \label{fig:06-eventWizardStep-5}
\end{figure}

\section{Seeding}
This part of the event wizard needs particular attention since the default is to use the American rating system which is quite different to the system used in the UK. 

\begin{figure}[!ht]
 \centering
 \begin{minipage}{0.4\textwidth}
  \centering
  \includegraphics[width=5cm,keepaspectratio]{06-eventWizardStep-6}
  \caption{Event Wizard -- Step 6, Initial Seeding Method} \label{fig:06-eventWizardStep-6}
 \end{minipage}
 \hfill
 \begin{minipage}{0.4\textwidth}
  \centering
  \includegraphics[width=5cm,keepaspectratio]{06-eventWizardStep-6a}
  \caption{Event Wizard -- Step 6a, Changing the Seeding Method} \label{fig:06-eventWizardStep-6a}
 \end{minipage}
\end{figure}

When you are prompted to use the default seeding method shown in figure \ref{fig:06-eventWizardStep-6} you should click the \button{Change} button. This will produce a new window (figure \ref{fig:06-eventWizardStep-6b}) on which you should choose to rank fencers on a points list only.

\begin{figure}[!ht]
 \centering
 \begin{minipage}{0.4\textwidth}
  \centering
  \includegraphics[width=5cm,keepaspectratio]{06-eventWizardStep-6b}
  \caption{Event Wizard -- Step 6b, Seeding from a Points List} \label{fig:06-eventWizardStep-6b}
 \end{minipage}
 \hfill
 \begin{minipage}{0.4\textwidth}
  \centering
  \includegraphics[width=5cm,keepaspectratio]{06-eventWizardStep-6c}
  \caption{Event Wizard -- Step 6c, Choosing the Points List} \label{fig:06-eventWizardStep-6c}
 \end{minipage}
\end{figure}

\begin{figure}[!ht]
 \centering
 \includegraphics[width=5cm,keepaspectratio]{06-eventWizardStep-6d}
 \caption{Event Wizard -- Step 6d, Confirming the Points List} \label{fig:06-eventWizardStep-6d}
\end{figure}

On the next screen (figure \ref{fig:06-eventWizardStep-6c}) you should select the points list for the tournament.

Once you have clicked the \button{Next} button you will get the confirmatory screen shown in figure \ref{fig:06-eventWizardStep-6d}.

Clicking on the \button{Finish} button will almost complete setting the tournament ranking scheme. You will be again prompted for the initial ranking scheme shown in figure \ref{fig:06-eventWizardStep-6}, but this time with the method that you have chosen for ranking. Click on the \button{Next} button to complete the event wizard.

\section{Completing the Event Wizard}

The final two steps of the Event Wizard are only applicable to American tournaments. When prompted as to whether the event is a qualifier then you should select the \texttt{No, this is a not a qualifier option}, and the \texttt{Yes, this event is sanctioned} option should be chosen for the sanction type.

\begin{figure}[!ht]
 \centering
 \begin{minipage}{0.4\textwidth}
  \centering
  \includegraphics[width=5cm,keepaspectratio]{06-eventWizardStep-7}
  \caption{Event Wizard -- Step 7, Event Qualification} \label{fig:06-eventWizardStep-7}
 \end{minipage}
 \hfill
 \begin{minipage}{0.4\textwidth}
  \centering
  \includegraphics[width=5cm,keepaspectratio]{06-eventWizardStep-8}
  \caption{Event Wizard -- Step 8, Event Sanction} \label{fig:06-eventWizardStep-8}
 \end{minipage}
\end{figure}

Clicking on the \button{Finish} button will complete the wizard for this event and will take you to the main tournament screen shown in figure \ref{fig:06-mainTournamentWindow}. If you have further events to add to the tournament then you can repeat the event wizard for these by clicking on \button{New Event} on the left hand panel of the screen.

\begin{figure}[!ht]
 \centering
 \includegraphics[width=10cm,keepaspectratio]{06-mainTournamentWindow}
 \caption{The Main Competition Window} \label{fig:06-mainTournamentWindow}
\end{figure}

\section{Adding Competitors}
At this point you should now have all your competitions set up. You should also have a database that contains all the fencers who will be taking part in your tournament (\S\ref{sec:fencers}). You may also have imported a list of up to date rankings (\S\ref{sec:rankings}) though this may be done later. What you have not yet done is selected the fencers from the database who will be fencing in each competition. 

There are two ways of doing this, either by searching the database for matching records or by importing a CSV file, the latter is the easiest option and will be dealt with first.

\subsection{Importing Entries}

If you have used an online entry system and it can export CSV files then the easiest way of registering fencers for competitions is to import the CSV file. An example of such a file is shown in figure \ref{fig:06-registrationFile}\footnote{This is the from the DT Register application used in the Joomla! open source content management system (CMS) for publishing content on the World Wide Web}.

\begin{figure}[!ht]
 \begin{verbatim}
DT UserId,firstname,lastname,payment_type,paid,BFA_Number,Club,Age_Group
862,Bob,McCoig,PayPal,Paid,367862,Liverpool Fencing Club,Senior
852,Chris,Sheen,PayPal,Paid,563422,University of Leeds,Senior
850,Dave,Lewis,PayPal,Paid,180538,,Cadet
849,Buzz,Lee,PayPal,Paid,277454,University of York,Senior
844,Paul,Salisbury,Mail in Payment,Not Paid,219962,Fighting Fit,Senior
836,Charles,Wilkinson,PayPal,Paid,613699,University of Kent,Senior
829,Peter,Worsley,Pay at the Door,Not Paid,838728,West Fife,Senior
828,Martin,Clarke,Pay at the Door,Not Paid,297690,Sheffield Swords,Senior
819,Terry,Regan,Mail in Payment,Paid,303008,Barnsley Fencing Club,Cadet
815,Peter,Lewis,Mail in Payment,Paid,641773,Salle Kiss,Veteran
 \end{verbatim}
 \caption{A Typical Extract from a Registration CSV File}\label{fig:06-registrationFile}
\end{figure}

The only thing that the file must contain is a method of identifying the fencers, \fencingtime{} will match on surname, forename and nickname. These will be standard for registration applications but may not be unique, if you can include the BFA number, in your registration form then this does provide a unique key for matching fencers. 

Click on the \button{Competitors}  button on the main competition screen (figure \ref{fig:06-mainTournamentWindow}) with the appropriate competition selected from the left hand panel of the screen. This will give you the screen shown in figure \ref{fig:06-tournamentCompetitors}.

\begin{figure}[!ht]
 \centering
 \includegraphics[width=10cm,keepaspectratio]{06-tournamentCompetitors}
 \caption{The Tournament Competitors Screen} \label{fig:06-tournamentCompetitors}
\end{figure}

You can now click on the \button{Import} button to import from your CSV file. As before you will be prompted to enter the name of the CSV file, once you have specified this then you need to identify the fields in the file and match them to those \fencingtime{} uses. For the file shown in figure \ref{fig:06-registrationFile} the fields are matched as shown below (figure \ref{fig:06-tournamentCompetitorsImport}).

\begin{figure}[!ht]
 \centering
 \includegraphics[width=6cm,keepaspectratio]{06-tournamentCompetitorsImport}
 \caption{Matching the Fields in the CSV File} \label{fig:06-tournamentCompetitorsImport}
\end{figure}

Once the data is imported the window should now contain the competitors for the tournament (figure \ref{fig:06-tournamentCompetitorsImported}).

\begin{figure}[!ht]
 \centering
 \includegraphics[width=10cm,keepaspectratio]{06-tournamentCompetitorsImported}
 \caption{List of Imported Fencers} \label{fig:06-tournamentCompetitorsImported}
\end{figure}

As the data is imported a log will be produced showing any errors during the import process. If the imported data is in conflict with the database (e.g. different club affiliations in the two sources, or different years of birth) then you will be prompted to say which version is correct.

Even if there are no errors during the import you should not assume that the data is correct. You must validate the entries before starting the competition. This is dealt with in the section on validation (\S\ref{sec:validation}).

\subsection{Entering Competitors from the Fencer Search Form}
If you only have a small number of fencers to add to the competition then you can add these individually. On the \ref{fig:06-tournamentCompetitors} screen click on \button{Add Competitors}. This will produce the search form shown in figure \ref{fig:06-addFencersToEvent}. You can search by a variety of criteria though the Last name and Member \# (BFA Number) will probably be the most common. Enter your search criteria and click the \button{Search} button. The list of fencers matching the search criterion will be returned, if the fencer to be entered is part of the list then simply click on the check box to the left of entry to select one or more fencers and then the \button{Add Checked} button at the bottom of the screen.

\begin{figure}[!ht]
 \centering
 \includegraphics[width=10cm,keepaspectratio]{06-addFencersToEvent}
 \caption{The Fencer Search Window} \label{fig:06-addFencersToEvent}
\end{figure}

If the search returns no matches then you will need to add a new fencer. Click on the \button{New Fencer} button, this will produce the dialogue shown in figure \ref{fig:04-newFencer}, fill in the fields and press the \button{OK} to save the fencer to the database. You will need to ensure that the fencer is credited with any ranking points by clicking on the \button{Point Rankings} button.

\begin{figure}[!ht]
 \centering
 \includegraphics[width=6cm,keepaspectratio]{06-setRankingPosition}
 \caption{Setting the Ranking Position} \label{fig:06-setRankingPosition}
\end{figure}

Note that you can only set the ranking position for a fencer, not the the number of points that they have earned.

\subsection{Validating the Entry List}\label{sec:validation}

You should not assume that the list of competitors you have for the competition is correct. It should be validated against the entries you have from paper or online entries. In particular you should check that:

\begin{itemize}
 \item The number of entries in \fencingtime{} matches the number of entries in your paper or online entries
 \item The club affiliation that a fencer has in the \fencingtime{} database matches that on the entry form, fencers can and do change affiliation.
 \item If you are running an age group competition then you should check on the eligibility of fencers
 \item By default \fencingtime{} assumes that fencers are male, you should ensure that the fencers for women's competitions have a correct gender assigned to them. Checking gender more generally might be useful where the forename could be used by both males and females, e.g. “Alex”.
\end{itemize}

\subsection{Checking Against the BFA Membership Database}

One thing that can speed up the check in process is pre-validation of BFA membership, this can be done online. You should apply to the BFA head office for competition organisers access to the membership database. Once you have this then you will need to prepare a CSV file whose first field is the BFA number of competitors. This can be uploaded on the BFA web site and will return a CSV file containing the BFA number, the name that this corresponds to, the membership type and the expiration date. 

You should note any missing numbers, or invalid or expired membership in the \texttt{Registration Notes} for competitors. 

\chapter{Starting the Tournament}

\section{Check In List}

Before the tournament starts you must provide the people on the check in desk with a list of people who have entered together with information that may require to be checked or changed (e.g. BFA number, club affiliation or country, whether the fencer has paid or not).
You will be running the tournament with \fencingtime{}, you should not use alternative sources of data since these may be incompatible with that held by \fencingtime{}.

To print the check in list adjust the list of fields you wish to have on the list by clicking on the right mouse button while the pointer is in the title bar on the list of competitors (figure \ref{fig:06-tournamentCompetitorsImported}), then press the \button{Print} button. 

\section{Processing the Check In List}

The people on the check in desk must make sure that when they mark the check in sheet they do so with absolute clarity. However you can assist them by making the sheet as easy as possible to use:

When you are setting up the list of fields make sure you include the \texttt{Here box} by using the right mouse button on the title bar of the list of competitors. You should also have an understanding of what goes in the box. It is suggested that rather than ticks or crosses, which can be confused, you use \texttt{P} and \texttt{S} for “present” and “scratched” respectively. If the check out time has expired and a fencer has not checked in the mark the box \texttt{A} for “absent”. Do not leave any of the boxes empty since this is ambiguous.

If the desk has to collect money from fencers then include the \texttt{Paid} field. Since it is not possible to import this from a CSV file and the numbers who have not paid is likely to be small it is suggested that you only set the amount in the field of fencers who have yet to pay.

If membership numbers have been checked in advance and only a few are outstanding then include the fact that particular fencers need to be checked in the \texttt{Registration notes} field.

Once the check in has closed an announcement should be made giving the list of fencers who have not checked in (marked “absent”) and a limited amount of time to present themselves. After this the list should be processed. 

\begin{itemize}
 \item  Fencers who are absent or scratched should be removed from the list of competitors. This is done by clicking on the check box on the list of fencers (figure \ref{fig:06-tournamentCompetitorsImported}) and, once all the absent fencers have been ticked, then clicking on the \button{Remove Checked} button.
 \item You should any additional fencers who  been allowed a late entry by the check in desk. Ensure that their seeding is correct either from the database or from the current ranking lists.
 \item You should alter any club affiliations noted by the check in desk, this is done by clicking on the edit button (the icon with the pencil next to each fencer in figure \ref{fig:06-tournamentCompetitorsImported}), this will take you to the edit fencer dialogue (figure \ref{fig:04-newFencer})
 \item If the check in desk have collected any missing BFA numbers then you should also associate these with the appropriate fencer, these will then go into the database for the next tournament.
 \item Finally you should ensure that the number of fencers present in the check in list is the same as that in the list of competitors.
\end{itemize}

\section{Assigning Referees}
At this point you should have at least a provisional list of the referees who will be active during the competition. If you are going to use \fencingtime{} to control referee activity now is the time to add the list to your tournament.

Click on the \button{Referees} button which is underneath the event panel in the main event window (figure \ref{fig:03-mainScreen}). This will produce the dialogue box shown in figure \ref{fig:06-addRefereesToTournament}. Fill in names or select ratings and then click on the \button{Search} button to find referees matching your search terms. Click on the check box next to their names to select which ones you wish to use and then click on the \button{Add Checked} button to add them to the tournament.

You can also use the \button{New Referee} button to create a new referee. This will present you with the new referee window shown in figure \ref{fig:04-newRefereeScreen}. Once you have filled in the details then clicking on the \button{OK} will close this window and allow you to add your new referee to the tournament.  

\begin{figure}[!ht]
 \centering
 \includegraphics[width=10cm,keepaspectratio]{06-addRefereesToTournament}
 \caption{Adding Referees to the Tournament} \label{fig:06-addRefereesToTournament}
\end{figure}

\section{Starting the Event}
Once you have corrected any errors and omissions you can start the event. Click on the \button{Start Event} button on the main competition screen (\ref{fig:06-mainTournamentWindow}) and you will be presented with the initial seeding window shown in figure \ref{fig:07-eventStartSeedings}.

\begin{figure}[!ht]
 \centering
 \includegraphics[width=10cm,keepaspectratio]{07-eventStartSeedings}
 \caption{Initial Seedings} \label{fig:07-eventStartSeedings}
\end{figure}

You should rarely need to alter this since you already have the seeding in the database, however if you do need to make modifications then checking the \texttt{Allow unrestricted manual seeding} box will allow you to use the arrows at the side of the window to move fencers up or down the list.

You should publish the list of fencers in at least two places in the hall and announce that this has been done. You should allow a few minutes for them to check that they are on the list, have the correct affiliation and ranking.

Once you are satisfied with the list then click on the \button{OK} button to start the first round.

\chapter{Poule Rounds}

\section{Creating the Poules}

At the start of a round you will be presented with the screen shown in figure \ref{fig:08-eventStartPouleRound}, you should select the \texttt{Pools} option and click the \button{OK} button. 

\begin{figure}[!ht]
 \centering
 \begin{minipage}{0.4\textwidth}
  \centering
  \includegraphics[width=5cm,keepaspectratio]{08-eventStartPouleRound}
  \caption{Choosing a Poule Round} \label{fig:08-eventStartPouleRound}
 \end{minipage}
 \hfill
 \begin{minipage}{0.4\textwidth}
   \centering
   \includegraphics[width=5cm,keepaspectratio]{08-eventStartPoules}
  \caption{Choosing the Poule Sizes} \label{fig:08-eventStartPoules}
 \end{minipage}
\end{figure}

If this is the first round of poules the next screen (figure \ref{fig:08-eventStartPoules}) will give you options for the number of poules and their sizes. 

Normally you should attempt to run with poules of 6 or 7 in size. When you click on the \button{OK} button you will be given a screen which gives you the composition of the poules.

\begin{figure}[!ht]
 \centering
 \includegraphics[width=10cm,keepaspectratio]{08-eventStartPouleAllocations}
 \caption{Poule Allocations} \label{fig:08-eventStartPouleAllocations}
\end{figure}

\fencingtime{} will attempt to produce poules which:

\begin{enumerate}
 \item minimise conflicts, e.g. too many or too few competitors from a particular club in a poule or fencers with the same surname in the same poule.
 \item Have approximately the same strength based upon the rankings.
\end{enumerate}

If this is not possible then a warning icon will be shown. 

You can move fencers between poules by simply dragging the name from one poule and dropping them in another. Alternatively right-clicking on the surname, this will show the window illustrated in figure \ref{fig:08-moveFencerInPoule}. You can either move the fencer to another poule or swap them with a fencer in another poule. To do the former simply select the poule from the drop down list and press the \button{OK} button, to do the latter select the poule which contains the fencer you wish to swap with and then the fencer and click \button{OK}.

\begin{figure}[!ht]
 \centering
 \includegraphics[width=6cm,keepaspectratio]{08-moveFencerInPoule}
 \caption{Moving a Fencer Between Poules} \label{fig:08-moveFencerInPoule}
\end{figure}

\section{Assigning Referees}

If you are using \fencingtime{} to allocate referees then you should assign the referees to the poules before you send the score sheet out. There are two ways to do this:
\begin{enumerate}
 \item Manually -- Below each poule there is a drop down from which you can select either one or two referees per poule.
 \item Automatically -- Clicking on the \button{Assign Refs to Pools} button below the list of poules will assign all the referees taking into account the affiliation of the fencers in the poules and the affiliation of the referee.
\end{enumerate}

\section{Entering Piste Numbers}

Once you are satisfied with the poule layout then you can enter the piste numbers. On the poule allocations screen (figure \ref{fig:08-eventStartPouleAllocations}) simply type in the number into the \texttt{Strip \# entry} field. Once you have done this click on the \button{Start Pool Round} button to show the poule information (figure \ref{fig:08-eventStartPouleSheets}).

\begin{figure}[!ht]
 \centering
 \includegraphics[width=10cm,keepaspectratio]{08-eventStartPouleSheets}
 \caption{Poule Sheets} \label{fig:08-eventStartPouleSheets}
\end{figure}

\section{Announcing the Poules}

Once you have created the poules you can print an alphabetical list of fencers and piste assignments by clicking on the \button{Pool Assignments} button and selecting the \texttt{Competitor} order tab. 

These sheets should be pinned up in at least two different locations and an announcement made that this has been done. This is much more efficient than attempting to read the assignments out over a PA system or by standing in the middle of the hall and shouting.

You should allow a few minutes for people to check the assignments and come to the DT if they have not been included. 

You can then print the poule sheets by clicking on the \button{Print Pools} button.

\section{Adding a Late Comer to the Poules}\label{sec:pouleLatecomer}

\fencingtime{} will allow a late comer to be added to a poule. On the poule information screen (figure \ref{fig:08-eventStartPouleSheets}) select the poule to which you want add the fencer and click on \button{Add Latecomer}. This will give you the standard fencer search screen (figure \ref{fig:04-fencerDatabase}), here you can search for a fencer by name, membership number or other criteria. Select the fencer by clicking on the check box and then the \button{Add Checked} button. Once you have done this you will need to reprint the poule sheet. 

\section{Redrawing the Pools}

At international level the practice is not to redraw the poules regardless of whether a mistake has been made. However FIE competitions are much more regulated than domestic competitions and you may have to make waive this unwritten rule, especially if the mistake was made by the DT.

\subsection{Fencer Left out of the Pools}

If you have left a fencer out of the poules then you must add the fencer to the competition. This can be done using the same process as adding a late comer (\S\ref{sec:pouleLatecomer}). If you have poules of 6 and 7 then you should add the fencer to a poule of 6 which does not lead to a poule with a significantly different strength to the other poules.

If you only have poules of 7 then you will have to redraw.

If the fencer who was left out has a significant ranking (e.g. you omitted the number 2 seed) then it will almost certainly prove impossible to place the fencer in a poule without unbalancing the strength of the poules. In this case you should redraw the poules.

\subsection{Fencer Included Who is Absent}

If you have included a fencer who is absent then the action you take should depends on which poule they were in and the ranking of the fencer. The basic principle is to ensure fairness to other fencers in the competition.

If you have mixed poules of 6 and 7 and the fencer was in a poule of 7 and had a low rank then you can safely remove them from the poule and reprint the poule sheet.

If the fencer was in a poule of 6 or had a high rank then you should consider a redraw. 

\subsection{Incorrect Draw due to Ranking}

If you have correctly ranked the fencers in the competition then you should not redraw. 

If you have made a mistake in the rankings then you should only consider a redraw if there is a significant difference between the ones you have used and the published ones.

You should include the rankings on the list of fencers that you display in the hall.

Note that if you have used ranking points to \gls{seed} the fencers and the positions include one or more foreign fencer then fencers below the foreign fencers will be pushed down in position, though the relative positions will remain the same. You may need to explain this to fencers who wish to know why their position is not the same as on the ranking list.  

\subsection{Incorrect Draw due to Affiliation}

If the fencer has not provided the correct affiliation details when entering the tournament or during check in then the draw should stand. 
You may have to explain why you have modified the fencer's affiliation, e.g. some fencers may have entered using a full club name while others have used initials and you have regularised the entries.

If you have made a mistake then you should redraw only if it makes a significant difference to the competition.

\section{Entering the Results of the Pools}

As the poule sheets are returned you can enter the results by selecting the poule from the left hand panel of the main poule screen (figure \ref{fig:08-eventStartPouleSheets}). Click in the first empty box on the top line and type the value from the poule sheet. As you type the number \fencingtime{} will automatically move you to the next box, there is no need to press the return key to move to the next line.

If you have a bout that has gone to time and not reached the full score then simply type the number of hits, \fencingtime{} will work out whether, for example a “3” is a win or a loss. Once you have completed the sheet then the \gls{indicators} will be calculated automatically, as shown in figure .

\begin{figure}[!ht]
 \centering
 \includegraphics[width=10cm,keepaspectratio]{08-eventPouleSheetCompleted}
 \caption{A Completed Poule Sheet} \label{fig:08-eventPouleSheetCompleted}
\end{figure}

If a fencer has not been able to complete all their fights you should withdraw them, none of their results will then be counted. Click on the \button{Withdraw/Exclude} button and select the fencer to withdraw. If the fencer failed to complete the poule because of medical reasons then you should withdraw them, if they have been black carded then you should exclude them.

You should validate that the entries on the screen match those on the poule sheet. To avoid simply scanning the sheets it is recommended that you read the entries downwards instead of across. Once you have validated the results then mark the printed sheet in some way to show that it has been processed. You do not need to note the time on the sheet since \fencingtime{} keeps timestamps for significant events, of which the completion of the poule sheet is one.

\section{Errors on the Poule Sheet}

During data entry \fencingtime{} may produce a warning message informing you that there is an error on the score, usually two fencers being credited with a victory or with equal scores in a particular match. The first thing you should do is to make sure that you have entered the data correctly from the boxes on the sheet. If you have done so then check the score for the match on the bottom of the sheet.

If you still cannot reconcile the scores then this almost certainly means that there has been more than one transposition. Enter the remainder of the data, this will allow you to determine how many fencers (and referees) you need to call to the DT.

\section{Printing the Results of the Pools}

Once the poules have been completed and all the results entered you should print copies which can be displayed for fencers to check. You should allow a specific amount of time for this to happen before proceeding to the next round of the tournament.

If there are any disputes then you should initially check whether a mistake was made during data input. If it was then this can be corrected on the poule entry window (figure \ref{fig:08-eventPouleSheetCompleted}) for the appropriate poule. If it was not then both fencers should be brought to the DT to resolve the issue. If it cannot be resolved then the result should be allowed to stand.

\section{Abandoning After the Pools, but Before the Direct Elimination}

If a fencer tells you they wish to abandon after completing their poule fights then there are two possible alternatives:

\begin{enumerate}
 \item If they abandon after the poules then \fencingtime{} will mark them as \texttt{DNF} (Did Not Fence) and place them last in the rankings.
 \item If they proceed to the direct elimination round and then tell their opponent and assigned referee that they wish to withdraw then \fencingtime{} will award the bout to their opponent but they will retain their position in the final ranking table as though they had lost the bout.
\end{enumerate}

You should explain the position to the two scenarios to the fencer.

If they do not wish to be included in the DE then you should select the poule that they were in and click on the \button{Withdraw/Exclude} button, select their name and withdraw them.

If they wish to take the second option then draw the DE and advise the fencer's opponent of the situation and that he or she has a bye. Strictly the fencer should present themselves at the piste, but there is no actual need to do so.

\section{Withdraw or Exclude}

\fencingtime{} has two alternatives for eliminating a fencer from a tournament:
\begin{description}
 \item[Withdraw (or abandon)] is used when the fencer is unable to continue for medical or other reasons
 \item[Exclude] is when the fencer receives a \gls{black card}
\end{description}

In the case of the poules \fencingtime{} will retain the fencer in the ranking list but place them last, marking them as \texttt{DNF} (Did Not Fence). If they are excluded then an entry will be placed last in the ranking list, but the name will appear as Fencer Excluded.

The situation is slightly different in the DE. In both cases the fencer effectively loses the fight and their opponent wins, the result of the bout will be used in the calculation of the final rankings. If the fencer has withdrawn then they will be credited with their position in the rankings, if they were excluded then their name will replaced by Fencer Excluded in the ranking list.

\section{Ranking After the Poules}

Once the poules are finished and any challenges to them have been reconciled then you can print the ranking list. You can either click the \button{Start Next Round} button or use the \button{Turn DE Wizard} once the tableau has been generated. Using the former will show the dialogue window illustrated in figure \ref{fig:08-eventPoulePromotion}. Clicking on the \button{Print} button will print the ranking list which you should display with the rest of the paperwork for the event.

\begin{figure}[!ht]
 \centering
 \includegraphics[width=6cm,keepaspectratio]{08-eventPoulePromotion}
 \caption{Ranking After the Poules} \label{fig:08-eventPoulePromotion}
\end{figure}

\chapter{Direct Elimination}
Once you have printed the ranking list and decided on how many fencers you wish to promote to the next round you should click on the \button{OK} button.  This will prompt you as to what kind of round you want, either a poule round or a direct elimination round. Select \texttt{Direct Elimination} as shown in figure \ref{fig:09-eventStartDE} and the tableau will be created automatically. 

\begin{figure}[!ht]
 \centering
 \begin{minipage}{0.4\textwidth}
  \centering
  \includegraphics[width=5cm,keepaspectratio]{09-eventStartDE}
  \caption{Selecting a DE Round} \label{fig:09-eventStartDE}
 \end{minipage}
 \hfill
  \begin{minipage}{0.4\textwidth}
  \centering
  \includegraphics[width=5cm,keepaspectratio]{09-saveTournament}
  \caption{Saving the Tournament} \label{fig:09-saveTournament}
 \end{minipage}
\end{figure}

Note that you will also be asked whether you wish to save the tournament at this point (figure \ref{fig:09-saveTournament}), whenever this window appears you should always respond in the affirmative. This will allow you to correct any mistakes that you inadvertently make.
      
\section{Entering Piste Numbers and Times}

A typical tableau is shown in figure \ref{fig:09-eventInitialTableau}.

\begin{figure}[!ht]
 \centering
 \includegraphics[width=12cm,keepaspectratio]{09-eventInitialTableau}
 \caption{The Initial Tableau} \label{fig:09-eventInitialTableau}
\end{figure}
 
Next to each bout are two icons, one above the line and one below. Clicking on the icon below the line (the one that looks like a hammer) will allow you to enter the piste number and the time for the bout, as shown in figure \ref{fig:09-deTimeAndPiste}.

\begin{figure}[!ht]
 \centering
 \includegraphics[width=5cm,keepaspectratio]{09-deTimeAndPiste}
 \caption{The Initial Tableau} \label{fig:09-deTimeAndPiste}
\end{figure}

You should set the times and pistes for at least the current round. If you are confident about the timing of the competition then you can also set the information for subsequent rounds based on the timing estimates in table \ref{tab:05-estimatedTimings}.

\section{The Turn DE Wizard}

The easiest way to print the tableau and other information is to use the \button{Turn DE Wizard} button. This will give the dialogue shown in figure \ref{fig:09-eventTurnDEwizard}.

\begin{figure}[!ht]
 \centering
 \includegraphics[width=6cm,keepaspectratio]{09-eventTurnDEwizard}
 \caption{The Turn DE Wizard} \label{fig:09-eventTurnDEwizard}
\end{figure}

Check the boxes for the information that you require printed. The items in steps 2 and 4 are the most likely for an average tournament while those in step 3 will be used for larger tournaments with a more formal assignment of referees.

\section{Printing the Score Sheets}

If you do not use the turn DE wizard then score sheets for matches can be printed by clicking on the \button{Print Bouts} button in the tool bar for the tableau (figure \ref{fig:09-eventInitialTableau}). This will present you with the dialogue window shown in figure \ref{fig:09-printDEsheets}. If you are running your competition round by round, e.g. all of the last 64 matches, then all of the last 32 matches then ensure that you deselect the option to print bouts where only one competitor has been determined. 

\begin{figure}[!ht]
 \centering
 \includegraphics[width=6cm,keepaspectratio]{09-printDEsheets}
 \caption{Selecting DE Score Sheets to Print} \label{fig:09-printDEsheets}
\end{figure}

In order to help in tracking the progress of the bouts you can click on the check box next to the \texttt{Ready} text at the side of each bout on the right hand side of the event window shown in figure \ref{fig:09-eventInitialTableau}. This will change the status to \texttt{Fencing} and change the colour so as to make it obvious which bouts are in progress.

\section{Assigning Referees}

Note that referees cannot be assigned either manually or automatically to DE bouts by \fencingtime{}. The only method of assignment is by writing the name of the referee on the score sheet. 

\section{Entering the Results}

There are three principle ways of selecting a match to score:
\begin{enumerate}
 \item Click on the icon above the line next to the match 
 \item Double click on the match in the Active Bouts panel to the right hand side of the tableau 
 \item Enter the bout id printed on the score sheet in the Score bout ID and press the return key
\end{enumerate}

In each case you will be presented with a scoring box shown in figure \ref{fig:09-deScoreEntry}.

\begin{figure}[!ht]
 \centering
 \includegraphics[width=5cm,keepaspectratio]{09-deScoreEntry}
 \caption{Scoring a DE Bout} \label{fig:09-deScoreEntry}
\end{figure}

Simply enter the scores and click on the \button{OK} button. The tableau will be updated and if the scoring of this match means that another match can be scheduled it will be added to the list of active bouts. 

If you are tracking referee activity with \fencingtime{} you should also use the \texttt{Referee} drop down to show who actually refereed the bout.

\section{Alternative Methods of Running the DE}

\subsection{The Single Tableau Method}

This is the method described above, you send out the score sheets for a particular phase of the DE (e.g. last 64) at the same time and do not start any more matches until all the score sheets have been returned.

The advantages of this is that it is easy to understand where you have reached and how long it should take to complete, it also saves on confusion as to which matches are actually out.

The disadvantage is that a hold-up one one piste can cause the rest of the tournament to be delayed. Inevitably this will occur at a phase of the competition where you are running four fights on this piste and recovery will then be difficult.

\subsection{Continuous Fencing}

In this method you issue score sheets as soon as there are matches in which the two fencers are determined. You can use this method if:

\begin{itemize}
 \item You try to ensure that the same piste is used for the next match in the tableau, so that the winning fencer does not have to move. This may not always be possible, you may have winning fencers from two pistes.
 \item You have a good PA system, or have instructed the winning fencers to remain by the piste on the completion of their matches
 \item You have briefed the referees to return each match as soon as it is complete or, better, ensure that your runners move from piste to piste collecting score sheets. 
\end{itemize}

You must allow fencers time after the completion of a bout before they start the next one. \fencingtime{} can assist in this since it will time down from the ten minutes allowed once the score sheet has been returned.

While this gets around the problem of a single piste holding up proceedings and a hiatus while a phase of the DE is scored it can be chaotic, with two or three different phases of the tableau taking place at the same time. The other difficulty is that it does not give your referees any chance of a break.

\subsection{In Quarters or Octants}

A variant of the continuous fencing method is to run the tableau in quarters or octants. 

If you look at the planning sheet in figure \ref{fig:05-tournamentPlanningSheet} you can see that the initial DE rounds for the men's foil are scheduled to run on 8 pistes. One can use a feature of \fencingtime{} which allows the tableau to be divided into quarters or octants. 

Click on \button{Assign Strips}, this will produce the window shown in figure \ref{fig:09-tableauPartition}.

\begin{figure}[!ht]
 \centering
 \includegraphics[width=10cm,keepaspectratio]{09-tableauPartition}
 \caption{Assigning Pistes in Octants} \label{fig:09-tableauPartition}
\end{figure}

Fill in the piste numbers and click on the \button{OK} button. \fencingtime{} will now update the tableau, effectively dividing into eight smaller and non-overlapping tableaux, an example of which is shown in figure \ref{fig:09-deInOctants} below.

\begin{figure}[!ht]
 \centering
 \includegraphics[width=10cm,keepaspectratio]{09-deInOctants}
 \caption{A Tableau Divided in Octants} \label{fig:09-deInOctants}
\end{figure}

The piste numbers are automatically assigned, all that is required is to set the start times for each bout. 

Running the direct elimination round in this way has a number of advantages:

\begin{enumerate}
 \item You can give the referee for the piste a copy of the tableau and a set of score sheets (use the \texttt{Print bouts where only one competitor has been determined} and \texttt{Print bouts where neither competitor has been determined} options when printing score sheets (figure \ref{fig:09-printDEsheets}). They can run all the bouts in their octant until only a single competitor remains. Once all the octants are completed then the fencers remaining constitute the quarter-finalists. If the tableau was partitioned into quarters rather than octants then one would have reached the semi-final stage.
 \item The fencers remain on the same pistes throughout their bouts, there is no need for them to move pistes with all the problems of them not knowing where they should be, or not turning up in time. You can use the \button{View Assignments} button on the \gls{strip} assignment dialogue (figure \ref{fig:09-eventInitialTableau}) to see an alphabetical list of fencers by piste and print the list so that the fencers know which piste to go to.
 \item It makes optimal use of referees.
\end{enumerate}

To avoid not knowing what stage a particular octant or quadrant is at and to enable you to print the tableau with results you should ensure that your runners collect the score sheets as soon as possible after matches are completed. You should also reissue your referees with the refreshed tableau.

Again, this means that the referees will be constantly active, you should therefore either have multiple referees per octant or make sure that referees are rotated at intervals.

\section{Which Method to Use}

There is no hard and fast rule for running the tableau, the method you use will depend on how many pistes you have, how many referees and how confident you are in keeping control of the competition. 

\section{Publishing Interim Results}

As each phase of the tableau is completed you should consider replacing the tableau that you have displayed with a refreshed one containing the results. At the same time you should publish the running list of results by clicking on the \button{Results So Far} button and printing out the results. 

\section{Completion of the Tableau}

Once the tableau is complete then you should click on the \button{Finish Event} button to complete the competition. This will allow you to print the reports including a set of final rankings. 

A window giving classifications will be shown when you click the \button{Finish Event} button. This is purely for American tournaments, you can safely click on the \button{OK} button at which point a table containing the final rankings will be displayed, as shown in figure \ref{fig:09-finalResults}.

\begin{figure}[!ht]
 \centering
 \includegraphics[width=10cm,keepaspectratio]{09-finalResults}
 \caption{The Final Results} \label{fig:09-finalResults}
\end{figure}

\chapter{After the Competition}

\section{Publishing Final Results}

\subsection{At the Venue}

The \button{Print} button on the final results (figure \ref{fig:09-finalResults}) can be used to print the final results for display in the appropriate areas. 

\subsection{On the Internet}

\fencingtime{} can export files in a variety of formats, if you want HTML for display on a web site then click on the \button{Export} button on the final results screen and select HTML as the format. This will produce a file that contains the final rankings, the results of the poules and the DE tableau. 

The HTML file is styled within the file itself, so if you are simply uploading this as a static page then there is no more to be done. 
If the results are to be loaded into a database then it is also possible to export the data as a CSV file.

\section{Creating Result Packs}

If you need to create a more comprehensive results pack, for example for FIE or EFC sites or for foreign teams then you may need to provide additional information. 

\begin{description}
 \item[Fencers] a list of fencers who were present can be exported by either using the \button{Competitors} button on the \texttt{Summary} tab for the tournament or by selecting \menu{View} $\rightarrow$ \menu{Reports} $\rightarrow$ \menu{Competitor List} menu item.
 \item[Results of Poules] these are included in the export of the final results
 \item[Ranking after the Poules] if you need this then you should ensure that you export it at the end of the poules using the \button{Export} button on the promoted fencers screen (figure \ref{fig:08-eventPoulePromotion})
 \item[Tableau] this is included in the export of the final results
 \item[Final Results] This is included in the export of the final results
 \item[Activity of Referees] this is available from the \texttt{Summary} tab for the competition.
\end{description}

If you need results in an FIE format then instead of exporting as HTML you should choose either Engarde FFF or FIE XML format, dependent on what is required.

\section{Who Needs the Results?}

This depends on the type of competition you are running.

\begin{enumerate}
 \item For a regional level \gls{event} then all you may require is a set of results for your regional committee and web master.
 \item For UK open competitions you will need to provide results for:
 \begin{description}
  \item[British Fencing] for publication on the web site
  \item[The Sword] for inclusion in the results pages
  \item[Weapon Ranking Coordinators] for calculation of the monthly rankings
 \end{description}
 \item For international tournaments you may need to provide results packs for each competing nation, for the FIE or EFC and for the media.
\end{enumerate}

\chapter{What to Do When it All Goes Wrong}

\section{Correcting an Earlier Mistake}

\subsection{In the Poules}

You can correct a mistake in the poules up to the time you start a new round. Click on the poule in question on the main poule screen (figure \ref{fig:08-eventPouleSheetCompleted}) and enter the new results. 

If you have followed the procedure for publishing results outlined earlier then you should refuse to correct any errors reported after the DE has started. It is the fencer who is responsible for checking that correct results were entered on the poule sheet and that the printed results are also correct.

\subsection{In the DE}

You can correct a mistake in the DE up to the time that you click on the \button{Finish} button to finish the tournament. Select the match that you wish to correct from the tableau display (figure \ref{fig:09-eventInitialTableau}) and enter the correct result.

If the tournament has been finished then you need to click on the \button{Unfinish Event} button, you can then correct a result in the same way as above.

\section{Restoring a Back Up}

At various times during the \gls{competition} \fencingtime{} will prompt you to take a back up of the tournament. You should always do so when prompted. The tournament data will normally be stored in a file in the \texttt{My Tournaments} directory. However it may be a good idea to store the data either on a network or USB drive so that if your computer fails during the tournament the information is still available.

In general you will not need to restore from a backup unless you are moving to the tournament to another computer. If, for example, there has been a major mistake in the poule round and you have started the DE round then simply click on the \button{Abort Round} button to abort the current round and revert to the previous one.

\section{Copying Data to Another Computer}

There are two processes for copying the data between computers:

\begin{itemize}
 \item Transferring the database
 \item Copying the tournament files.
\end{itemize}

\subsection{Transferring the Database}

Transferring a database is a two-step process. The first step is to export the \gls{database} from the “source” computer’s copy of \fencingtime{} so that it can be taken to the “destination” computer. To do this, choose the first option presented by the Transfer Database Wizard (running on the source computer’s copy of \fencingtime{}.) You will be asked to specify a folder where you want the transfer file placed.

After completing this step, \fencingtime{} will generate a file called \texttt{FencingTimeTransferData.ft3}. This file contains all of the information needed to bring the database to a new computer.

The next step is to copy the \texttt{FencingTimeTransferData.ft3} to the destination computer. You can do this by whatever method is most convenient, e.g. copy it to a USB drive or over the network. Be sure you note where you place the file on the new computer. The desktop is usually a good choice.

The next step requires that \fencingtime{} is already installed on the destination computer. If you have not already done so, install \fencingtime{} and then launch it. If you are asked whether you want to create or convert a database file, tell \fencingtime{} to create a new database. 

Next, use the \menu{Tools} $\rightarrow$ \menu{Move Master Database to/from Another Computer} option to launch the wizard again. This time, you will select the second option presented -- the option to import the database.

When prompted, select the folder that contains the \texttt{FencingTimeTransferData.ft3} that you brought to the destination computer from the source computer.

Once you have done this, \fencingtime{} will display a confirmation step before beginning the import process. It is important to understand that the imported database will completely replace all of the data on the destination computer. You will still be able to open tournament files, but any other fencer, club and referee data will be lost. 

Once you confirm the import, \fencingtime{} will replace your database with the one from the source computer. Once complete, you will have access to all of the data that was available on the source computer. 

It is recommended that you install \fencingtime{} on your second computer and do the transfer of the database before the tournament rather than during it. While the transfer is quick it is something that you should avoid doing during the fraught time of running a tournament.

\subsection{Copying the Tournament Files}

All that is necessary is to copy the \texttt{My Tournaments} tournament directory from one computer to the other, either using a USB drive or over the network.

\chapter{Other Reports}

\fencingtime{} offers some other useful reports:

\begin{itemize}
 \item On the main competition screen you will see a button \button{Mark Time}. Each significant event will have a time stamp attached to it. These include the start of a round, when score sheets went out, when score sheets came back and when the tournament finished. You can also add your own events. This is useful if you have disputes since it enables you to provide an audit of what happened during the tournament. It also serves as a planning tool for future years.
 \item The schedule for the day is available from the \menu{View Reports} menu, this gives the list of events and the times that they are due to run.
 \item The final result summary gives a list of the competitors in all the events. Each \gls{competitor} is listed with the event name, their final place, their \gls{club}, division, and country affiliations, and their classifications. The major use for this report, which is available as a CSV file, is as a source for a database upload for cumulative results over the years the tournament runs.
 \item The financial summary report simply tabulates the registration and entry fees.
\end{itemize}


\printglossary
\addcontentsline{toc}{chapter}{Glossary}

\end{document}











